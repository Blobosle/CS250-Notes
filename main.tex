\documentclass[11pt]{article}

%----------%
% Packages %
%----------%

\usepackage{tikz}
\usepackage{fancyhdr}
\usepackage{enumitem}
\usepackage[letterpaper, margin=1in]{geometry}
\usepackage{upquote}
\usepackage[T1]{fontenc}
\usepackage{textcomp}
\usepackage{listings}
\usepackage{color}
\usepackage{inconsolata}
\usepackage{graphicx}
\usepackage[parskip]{setspace}
\usepackage[most]{tcolorbox}
\usepackage{hyperref}
\usepackage{amssymb}
\usepackage{tabularray}
\usepackage{subcaption}

\usepackage{caption}
\captionsetup[lstlisting]{labelformat=empty,labelsep=none}

%-----------------%
% Header & footer %
%-----------------%

\pagestyle{fancy}
\lhead{Blobosle}
\rhead{CS 250}
\chead{Final (Notes)\\Spring 2025}
\cfoot{}
\rfoot{Page \thepage}

%-------%
% Title %
%-------%

\title{\textbf{CS 250: Computer Architecture\\Final Exam\\Spring 2025}}
\author{Benjamin Lobos Lertpunyaroj}
\date{\textit{May 8th, 10:30{\tiny AM} – 12:30{\tiny PM}}}

\setlength{\headsep}{3em}

%----------%
% Settings %
%----------%

\setlength{\parindent}{0pt}
\setstretch{1.5}

\definecolor{greenText}{rgb}{0.5, 0.7, 0.5}
\definecolor{greyText}{rgb}{0.5, 0.5, 0.5}
% \definecolor{codeFrame}{rgb}{0.5, 0.7, 0.5}
\definecolor{codeFrame}{rgb}{0.45, 0.66, 0.76}
\definecolor{moonstoneblue}{rgb}{0.45, 0.66, 0.76}
\definecolor{moondark}{rgb}{0.30, 0.50, 0.60}

\lstdefinestyle{code}{
    frame=single,
    rulecolor=\color{codeFrame},
    numbers=left,
    numbersep=8pt,
    numberstyle=\tiny\color{greyText},
    commentstyle=\color{greenText},
    basicstyle=\linespread{1.1}\ttfamily\footnotesize,
    keywordstyle=\ttfamily\footnotesize,
    showstringspaces=false,
    xleftmargin=1.95em,
    framexleftmargin=1.6em,
    breaklines=true,
    postbreak=\mbox{\textcolor{greenText}{$\hookrightarrow$}\space},
    % linewidth=0.6\textwidth,
}
\lstset{style=code, language=C}

%------–-------%
% Initial page %
%--------------%

\begin{document}

\maketitle

\vspace{1em}

\begin{center}
\section*{Exam contents and details for referencing}
\end{center}

\begin{itemize}[itemsep=-0.5em, left=0pt, label={•}]
    \item Final exam is held in Fowler Hall on May 8th (Thursday), from 10:30 {\tiny AM} to 12:30 {\tiny PM}.
    \item Previous cumulative book chapters
    \vspace{-0.8em}
    \begin{itemize}[itemsep=-0.5em, left=0pt, label={•}]
    \item Chapter 1 sections 1, 2, and 3.
    \item Chapter 2, sections 1, 2, 3, 4, 5, 6, and 7.
    \item Chapter 3, sections 1, 2, and 5.
    \item Chapter 4, sections 1, 2, 3, 4, 5, 6, 7, and 8.
    \item Chapter 5, sections 1, 2, 3, 4, 7, and 8.
    \item Chapter 8 (Appendix A), sections 1, 2, 3 (but not PLAs or ROMs), 5, 7 (lightly), and 8.
    \end{itemize}
    \item All lecture notes and lecture slides.
    \item All labs (1 - 11).
    \item The order of appearance of contents in this document is arbitrary.
\end{itemize}

\pagebreak

\section*{Appendix A}

\subsection*{Logic \& Gates}
An \textit{asserted} signal is logically true, the \textit{deasserted} is the opposite.

\begin{tcolorbox}[
    enhanced,
    attach boxed title to top left={xshift=6mm,yshift=-1.5mm},
    colback=moonstoneblue!20,
    colframe=moonstoneblue,
    colbacktitle=moonstoneblue,
    title=Two types of logic systems,
    fonttitle=\bfseries\color{white},
    boxed title style={size=small,colframe=moonstoneblue,sharp corners},
    sharp corners,
    label=box:logic-types,
]
    {\color{moondark}\textbf{Combinational logic}}: No memory in components, hence same output given same input. \\
    {\color{moondark}\textbf{Sequential logic}}: Memory in components, hence output depends on input and current memory state.
\end{tcolorbox}

\begin{figure}[htbp]
    \centering
    \fcolorbox{codeFrame}{white}{
        \includegraphics[width=0.5\linewidth]{img/gates.png}%
    }
    \caption{\textit{\texttt{AND} gate, \texttt{OR} gate, and inverter}}
\end{figure}

The gates can be combined to form different forms of logic. An example of this is $\overline{\overline{A} + B}$ which is equivalent to $A \cdot \overline{B}$ by De Morgan's law, seen in \autoref{fig:gates2}.

\begin{figure}[htbp]
    \centering
    \fcolorbox{codeFrame}{white}{
        \includegraphics[width=0.5\linewidth]{img/gates2.png}%
    }
    \caption{\textit{Logic gate implementation of example formula}}
    \label{fig:gates2}
\end{figure}

\subsection*{Decoders \& Multiplexors}

A \textbf{decoder} is a logic block that has an n-bit input and $2^n$ outputs, where there is one unique true bit as output from a unique set of bytes of input.

\begin{figure}[htbp]
    \centering
    \fcolorbox{codeFrame}{white}{
        \includegraphics[width=0.7\linewidth]{img/decoder1.png}
    }
    \caption{\textit{3-bit input decoder that generates $2^3=8$ different outputs (Out0 – Out7)}}
\end{figure}

$$2^n \text{ outputs} \ \therefore \ \log_2{(\text{output})} = \text{input bits}$$

Encoders are the other way around.

\textbf{Multiplexors} have a selector input (or control value), that will determine which inputs will become outputs.

In the case of the two-input MUX, its representation is the following, $C=(A \cdot \overline{S}) +(B \cdot S)$, using n (data inputs) AND gates, and one OR gate.

\begin{figure}[htbp]
    \centering
    \fcolorbox{codeFrame}{white}{
        \includegraphics[width=0.5\linewidth]{img/mux1.png}
    }
    \caption{\textit{Two-input multiplexor that generates one output depending on the selector input S}}
\end{figure}

\vspace{-2.5em}
$$n \text{ (data inputs)} \therefore \log_2{n} = S \text{ selector bits required to represent all inputs}$$

Often times a decoder generates n bits for a MUX, to be used as a selector signal.

\subsection*{Buses}

A collection of data lines that is treated as a single logical signal.

When showing a logic unit whose inputs and outputs are buses, the unit must be replicated a sufficient number of times to accommodate the width of the input.

You can use multiplexors to select between two buses, requiring n inputs to represent n-bit buses.

\begin{figure}[htbp]
    \centering
    \fcolorbox{codeFrame}{white}{
        \includegraphics[width=0.28\linewidth]{img/bus1.png}
    }
    \caption{\textit{1-bit multiplexors replicated 64 times to represent two 64-bit buses}}
\end{figure}

\subsection*{ALUs}

\begin{tcolorbox}[
    enhanced,
    attach boxed title to top left={xshift=6mm,yshift=-1.5mm},
    colback=moonstoneblue!20,
    colframe=moonstoneblue,
    colbacktitle=moonstoneblue,
    title=Operation done by the ALU,
    fonttitle=\bfseries\color{white},
    boxed title style={size=small,colframe=moonstoneblue,sharp corners},
    sharp corners,
]
    {\color{moondark}\textbf{Logic operations}}: \texttt{AND} and \texttt{OR} gate operations, with \texttt{NOR} being available through an inversion of both input signals with \texttt{AInvert} and \texttt{BInvert} control signals. \\
    {\color{moondark}\textbf{Arithmetic operations}}: Addition and subtraction through the full adder, and \texttt{BInvert} control signal on one input for determining the type of operation.
\end{tcolorbox}

The LEGv8 word is 64 bits wide, as such a 64 bit wide ALU is required (64 1-bit ALUs).

In its simplest form, a \underline{1-bit logical unit} for \texttt{AND} and \texttt{OR} operations simply requires a multiplexor an a one bit control signal to select between the two operations ($2^1=1$).

\begin{figure}[htbp]
    \centering
    \fcolorbox{codeFrame}{white}{
        \includegraphics[width=0.28\linewidth]{img/alu1.png}
    }
    \caption{\textit{1-bit logical unit for \texttt{AND} and \texttt{OR} operations}}
\end{figure}

Implementing addition requires two input operands, one output, a \texttt{CarryIn} bit carried from the less-significant bits of the operation (i.e. another 1-bit logical unit), and a \texttt{CarryOut} bit to be carried forward to the next more significant bit.

\begin{figure}[htbp]
    \centering
    \fcolorbox{codeFrame}{white}{
        \includegraphics[width=0.8\linewidth]{img/adder1.png}
    }
    \caption{\textit{Full adder that performs mod 2 addition}}
\end{figure}

The combination of the adder and the logic gates, coupled with a multiplexor with a control signal to determine the operation makes a complete 1-bit ALU, which can be seen in \autoref{fig:1bitalu}.
\pagebreak

\begin{figure}[htbp]
    \centering
    \fcolorbox{codeFrame}{white}{
        \includegraphics[width=0.2\linewidth]{img/alu2.png}
    }
    \caption{\textit{1-bit alu with logical operations and addition}}
    \label{fig:1bitalu}
\end{figure}

For expanding to a 64-bit ALU, the adders have to set up a ripple carry from the least to the most significant bit.

\begin{figure}[htbp]
    \centering
    \fcolorbox{codeFrame}{white}{
        \includegraphics[width=0.2\linewidth]{img/ripplecarry.png}
    }
    \caption{\textit{Ripple carry implemented for a 64-bit ALU}}
\end{figure}

By inverting the second input (\texttt{BInvert} = 1, seen in \autoref{fig:subalu}) and setting \texttt{CarryIn} to 1 in the least significant bit of the ALU, we get two’s complement subtraction of \texttt{b} from \texttt{a}.

To implement a \texttt{NOR} function, existing components can be combined, $(\overline{a+b})=\overline{a} \cdot \overline{b}$ (DeMorgan's theorem), which means we need an \texttt{AND} and two inverters for both \texttt{a} and \texttt{b}, seen in \autoref{fig:subalu}

\begin{figure}[htbp]
    \centering
    \fcolorbox{codeFrame}{white}{
        \includegraphics[width=0.32\linewidth]{img/alu3.png}
    }
    \caption{\textit{1-bit ALU that performs subtraction, and \texttt{NOR} operations}}
    \label{fig:subalu}
\end{figure}

On a 64-bit ALU we can use a zero flag to help with conditional branch instructions in LEGv8 (e.g. \texttt{CBZ}), as they receive to inputs and require to test if the subtraction has a zero.

The following represents this with an inversion of an \texttt{OR} tree on all results from the subtraction considering a 64-bit subtraction, fully represented in \autoref{fig:64aluzero}.
\vspace{-1em}
$$Zero = \overline{(R_0 + R_1 + R_2 + ... + R_{63})}$$
\vspace{-3em}

\begin{figure}[htbp]
    \centering
    \fcolorbox{codeFrame}{white}{
       \includegraphics[width=0.4\linewidth]{img/alu4.png}
    }
    \caption{\textit{64-bit ALU \texttt{OR} tree and an inverter for determining the Zero flag}}
    \label{fig:64aluzero}
\end{figure}

For a generalized symbol of the ALU, \autoref{fig:generalalu}, where \texttt{ALU operation} is the control signal of the MUX that determines the type of operation.

\begin{figure}[htbp]
    \centering
    \fcolorbox{codeFrame}{white}{
       \includegraphics[width=0.2\linewidth]{img/alu5.png}
    }
    \caption{\textit{General symbol for an ALU or an adder}}
    \label{fig:generalalu}
\end{figure}

\pagebreak

\subsection*{Clocks}

\begin{tcolorbox}[
    enhanced,
    attach boxed title to top left={xshift=6mm,yshift=-1.5mm},
    colback=moonstoneblue!20,
    colframe=moonstoneblue,
    colbacktitle=moonstoneblue,
    title=Clocking methodology semantics,
    fonttitle=\bfseries\color{white},
    boxed title style={size=small,colframe=moonstoneblue,sharp corners},
    sharp corners,
]
    {\color{moondark}\textbf{Edge triggered clocking}}: State changes occur on a clock edge. \\
    {\color{moondark}\textbf{Synchronous system}}: Type of memory system where data is read only when a clock signal indicates stability (i.e. non-changing value).
\end{tcolorbox}

A \hyperref[box:logic-types]{combinational logic block}, recieves an input and then generates an output for a state element which is updated on a clock edge.

An edge-triggered methodology allows a state element to be read and written in the same clock cycle without creating a race condition.

For this to work, the clock cycle must be long enough for the state element to have received a stable input before the next active clock edge.

\begin{figure}[htbp]
    \centering
    \fcolorbox{codeFrame}{white}{
       \includegraphics[width=0.5\linewidth]{img/clock1.png}
    }
    \caption{\textit{Edge-triggered state element to be read and written to in one active clock edge}}
\end{figure}

One such state element is the register file.

\subsection*{Flip-flops \& Latches}

\begin{tcolorbox}[
    enhanced,
    attach boxed title to top left={xshift=6mm,yshift=-1.5mm},
    colback=moonstoneblue!20,
    colframe=moonstoneblue,
    colbacktitle=moonstoneblue,
    title=Types of clocked memory elements,
    fonttitle=\bfseries\color{white},
    boxed title style={size=small,colframe=moonstoneblue,sharp corners},
    sharp corners,
]
    {\color{moondark}\textbf{Flip-flops}}: Edge-triggered element that changes the stored state only at a clock edge. \\
    {\color{moondark}\textbf{Latches}}: Level-sensitive element that changes the stored state at any time the clock is asserted.
\end{tcolorbox}

Flip-flops are build upon latches and are going to be used in edge-triggered systems.

A \textbf{D flip-flop} or \textbf{D latch} is used for storing the value of one data input signal, in the internal memory, at the clock edge.

To implement a \textbf{D latch}, it requires two inputs, the data to be stored \texttt{D}, and the clock signal \texttt{C}, producing two outputs, the value of the internal state \texttt{Q}, and its complement \(\overline{\mathtt{Q}}\).

The \hyperref[]{implementation} has cross-coupled \texttt{NOR} gates that store the state value unless \texttt{C} is asserted, in which case \texttt{D} replaces the value of \texttt{Q} and is stored.

\pagebreak

\begin{figure}[htbp]
    \centering
    \fcolorbox{codeFrame}{white}{
       \includegraphics[width=0.3\linewidth]{img/dlatch.png}
    }
    \caption{\textit{D latch, composed of crossed \texttt{NOR} gates and a \texttt{SR} latch}}
    \vspace{1em}
    \centering
    \fcolorbox{codeFrame}{white}{
       \includegraphics[width=0.35\linewidth]{img/dlatch1.png}
    }
    \caption{\textit{Progression of a D latch, assuming output is initially deasserted}}
\end{figure}

To implement a \textbf{D flip-flop}, with a \underline{falling-edge trigger}, we can use two D latches, master and slave. Master sets input \texttt{D} when \texttt{C} is asserted. When \texttt{C} falls, master is closed, but slave is open and gets its input from master's \texttt{Q}.

In this sense, the rising-edge represents when the master takes in the \texttt{D} value, and the falling-edge represents when the slave takes in the master's \texttt{D} producing the final \texttt{Q}.

\begin{figure}[htbp]
    \centering
    \fcolorbox{codeFrame}{white}{
       \includegraphics[width=0.5\linewidth]{img/dflip.png}
    }
    \caption{\textit{D flip-flop with a falling-edge trigger made from two D latches, master and slave}}
    \vspace{1.5em}
    \centering
    \fcolorbox{codeFrame}{white}{
       \includegraphics[width=0.5\linewidth]{img/dflip1.png}
    }
    \caption{\textit{Progression of a D flip-flop with a falling-edge trigger, where output is initially deasserted}}
\end{figure}

The minimum time \texttt{D} must retain a valid input is the setup time plus the hold time (after edge).

\subsection*{Register files}

A \textbf{register file} consists of a bunch of \textbf{registers} that can be read and written to, and a \texttt{WriteReg} control signal (clock).

For writing it requires the control signal, the number of the register to write to (\texttt{Write register}), and the data to write (\texttt{Write data}).

For reading it requires the numbers of the registers to read from (\texttt{Read register number 1 \& 2}), and it outputs the read contents from two registers (\texttt{Read data 1 \& 2}).

\begin{figure}[htbp]
    \centering
    \fcolorbox{codeFrame}{white}{
       \includegraphics[width=0.35\linewidth]{img/regfile1.png}
    }
    \caption{\textit{Register file with two read ports and one write port}}
\end{figure}

The implementation for the \underline{write port} consists of a decoder that will select one of the \texttt{n - 1} registers that will be \texttt{AND}ed with the \texttt{WriteReg} signal to act as the \texttt{C} input for the registers (D flip-flops). The \texttt{D} input for every register is the \texttt{Write data} input from the reg. file.

\begin{figure}[htbp]
    \centering
    \fcolorbox{codeFrame}{white}{
       \includegraphics[width=0.55\linewidth]{img/regwrite1.png}
    }
    \caption{\textit{Write implementation in the register file}}
\end{figure}

The implementation for the two \underline{read ports} consists of using the stored state of the registers (\texttt{Q} output), as inputs for two different MUXes that use the \texttt{Read register number 1 \& 2} reg. file inputs as control signals to output the information of the two registers specified.

\begin{figure}[htbp]
    \centering
    \fcolorbox{codeFrame}{white}{
       \includegraphics[width=0.55\linewidth]{img/regread1.png}
    }
    \caption{\textit{Read implementation in the register file}}
\end{figure}

\pagebreak

\section*{Chapter 2}

The size of a register in LEGv8 is 64 bits, which are denominated as doublewords (8 bytes).

A word, the natural unit of access in a computer, is 32 bits (4 bytes).

\subsection*{LEGv8 Assembly}

\begin{tcolorbox}[
    enhanced,
    attach boxed title to top left={xshift=6mm,yshift=-1.5mm},
    colback=moonstoneblue!20,
    colframe=moonstoneblue,
    colbacktitle=moonstoneblue,
    title=Relevant LEGv8 instructions,
    fonttitle=\bfseries\color{white},
    boxed title style={size=small,colframe=moonstoneblue,sharp corners},
    sharp corners,
]
    \begin{tabular}{@{} l @{\quad} l @{}}
    {\color{moondark}\textbf{Addition}}:           & \texttt{ADD X1, X2, X3}     \\
    {\color{moondark}\textbf{Subtraction}}:        & \texttt{SUB X1, X2, X3}     \\
    {\color{moondark}\textbf{Add immediate}}:      & \texttt{ADDI X1, X2, \#20}  \\
    {\color{moondark}\textbf{Subtract immediate}}: & \texttt{SUBI X1, X2, \#20}  \\
    {\color{moondark}\textbf{Load register}}:      & \texttt{LDUR X1, [X2, \#20] \ \ // Load mem. at addrs. X2 + 20} \\
    {\color{moondark}\textbf{Store register}}:     & \texttt{STUR X1, [X2, \#20] \ \ // Store at addrs. X2 + 20} \\
    {\color{moondark}\textbf{Logical AND}}:        & \texttt{AND X1, X2, X3}  \\
    {\color{moondark}\textbf{Logical Inclusive OR}}: & \texttt{ORR X1, X2, X3}  \\
    {\color{moondark}\textbf{Logical Exclusive OR}}: & \texttt{EOR X1, X2, X3}  \\
    {\color{moondark}\textbf{Bitwise Left}}:       & \texttt{LSL X1, X2, \#20}  \\
    {\color{moondark}\textbf{Bitwise Right}}:      & \texttt{LSR X1, X2, \#20}  \\
    {\color{moondark}\textbf{Conditional branch is 0}}: & \texttt{CBZ X1, L0 \ \ // Inmediate in operand is in word bytes} \\
    {\color{moondark}\textbf{Conditional branch not 0}}: & \texttt{CBNZ X1, L0 \ \ // Inmediate in operand is in word bytes} \\
    {\color{moondark}\textbf{Branch}}:             & \texttt{B L0 \ \ // Inmediate in operand is in word bytes}
    \end{tabular}
\end{tcolorbox}

The above list excludes instructions that set flags, e.g. \texttt{ADDS, SUBS, SUBIS, ADDIS}, and zero independent conditional branching, e.g. \texttt{B.cond}.

\subsection*{LEGv8 architecture design}

There are 32 64-bit registers, limited as larger number of registers implies increasing clock cycle time as electronic signals take longer as they must travel further. It also makes instruction formats more constrained.

Many architectures have alignment restriction that establish that words must start at addresses that are multiples of 4 and doublewords at multiples of 8.

\pagebreak

\subsection*{Signed and unsigned numbers}

\begin{tcolorbox}[
    enhanced,
    attach boxed title to top left={xshift=6mm,yshift=-1.5mm},
    colback=moonstoneblue!20,
    colframe=moonstoneblue,
    colbacktitle=moonstoneblue,
    title=Denomination of bits in a bit string,
    fonttitle=\bfseries\color{white},
    boxed title style={size=small,colframe=moonstoneblue,sharp corners},
    sharp corners,
    label=box:logic-types,
]
    {\color{moondark}\textbf{Least significant bit}}: The rightmost bit in a bit string. In a doubleword this is bit 63. \\
    {\color{moondark}\textbf{Most significant bit}}: The leftmost bit in a bit string. In a doubleword this is bit 0.
\end{tcolorbox}

A \textbf{sign and magnitude} representation of a signed number has one bit set aside to represent the sign of the integer. Issues come with having a \texttt{-0} and \texttt{+0}, and more steps needed to determine the sign by the adder.

For making the hardware simple, \textbf{two's complement} representation is used, which used leading \texttt{0}s to denote positives, and leading \texttt{1}s to denote negatives. This means that the most significant bit can be used as a sign bit, making conversions the following way.

\vspace{-1em}
$$(x_{63} \cdot -2^{63})+(x_{62} \cdot 2^{62})+(x_{61} \cdot 2^{61})+...+(x_{1} \cdot 2^{1})+(x_{0} \cdot 2^{0})$$

As an example,
\vspace{-1em}
$$11111111 \ 11111111 \ 11111111 \ 11111111 \ 11111111 \ 11111111 \ 11111111 \ 11111100_{\text{two}}=-4_{\text{ten}}$$

\underline{Overflow} occurs when the sign bit gets overridden, i.e. \texttt{0} when negative, \texttt{1} when positive.

Using two's complement, \underline{sign extension}, used for conditional branching, just requires to copy the sign bit \texttt{m - n} bits over, where \texttt{m} is the new size of the bit string.

\begin{tcolorbox}[
    enhanced,
    attach boxed title to top left={xshift=6mm,yshift=-1.5mm},
    colback=moonstoneblue!20,
    colframe=moonstoneblue,
    colbacktitle=moonstoneblue,
    title=Power of 2 prefix definitions,
    fonttitle=\bfseries\color{white},
    boxed title style={size=small,colframe=moonstoneblue,sharp corners},
    sharp corners,
    label=box:logic-types,
]
    {\color{moondark}\textbf{Kibibyte (Kib)}}: $2^{10}$ bytes. \\
    {\color{moondark}\textbf{Mebibyte (Mib)}}: $2^{20}$ bytes. \\
    {\color{moondark}\textbf{Gibibyte (Gib)}}: $2^{30}$ bytes. \\
    {\color{moondark}\textbf{Tebibyte (Tib)}}: $2^{40}$ bytes. \\
    {\color{moondark}\textbf{Pebibyte (Pib)}}: $2^{50}$ bytes.
\end{tcolorbox}

\pagebreak

\subsection*{Instruction format fields}

\begin{table}[htbp]
  \centering

  \begin{minipage}[t]{0.45\textwidth}
    \centering
    \begin{tabular}{|l|l|l|l|l|}
      \hline
      \texttt{Opcode} & \texttt{Rm} & \texttt{shamt} & \texttt{Rn} & \texttt{Rd} \\
      11 bits & 5 bits & 6 bits  & 5 bits  & 5 bits \\
      \hline
    \end{tabular}
    \captionof{table}{\textit{R‐type format, \texttt{ADD, SUB, LSL, LSR}}}
  \end{minipage}\hfill
  \begin{minipage}[t]{0.45\textwidth}
    \centering
    \begin{tabular}{|l|l|l|l|l|}
      \hline
      \texttt{Opcode} & \texttt{Address} & \texttt{Op2} & \texttt{Rn} & \texttt{Rt} \\
      11 bits & 9 bits & 2 bits  & 5 bits  & 5 bits \\
      \hline
    \end{tabular}
    \captionof{table}{\textit{D‐type format, \texttt{LDUR, STUR}}}
  \end{minipage}

  \vspace{1em}

  \begin{minipage}[t]{0.45\textwidth}
    \centering
    \begin{tabular}{|l|l|l|l|}
      \hline
      \texttt{Opcode} & \texttt{Immediate} & \texttt{Rn} & \texttt{Rd} \\
      10 bits & 12 bits & 5 bits  & 5 bits \\
      \hline
    \end{tabular}
    \captionof{table}{\textit{I‐type format, \texttt{ADDI, SUBI}}}
  \end{minipage}\hfill
  \begin{minipage}[t]{0.45\textwidth}
    \centering
    \begin{tabular}{|l|l|l|}
      \hline
      \texttt{Opcode} & \texttt{Address} & \texttt{Rd} \\
      8 bits & 19 bits & 5 bits \\
      \hline
    \end{tabular}
      \captionof{table}{\textit{Conditional branch format, \texttt{CB(N)Z}}}
  \end{minipage}

  \vspace{1em}

  \begin{minipage}[t]{0.5\textwidth}
    \centering
    \begin{tabular}{|l|l|}
      \hline
      \texttt{Opcode} & \texttt{Address} \\
      6 bits & 26 bits \\
      \hline
    \end{tabular}
    \captionof{table}{\textit{Unconditional branch format, \texttt{B}}}
  \end{minipage}

\end{table}

All instruction field formats are 32 bits, thus the name \textit{32-bit instructions}.

The \texttt{opcode} denotes the operation and format of an instruction.

The register operands, \texttt{Rn}, \texttt{Rm}, and \texttt{Rd} are 5 bits to represent the 32 registers ($\log_2{32}=5$).

The \texttt{address} field can represent a region of $\pm2^8$ bytes around the base register \texttt{Rn}.

The \texttt{shamt} field represents the shift amount used by bit-shift instructions.

\subsection*{Conditional Statement \& Loops}

\begin{lstlisting}[caption={Simple conditional statement implementation}]
    CBNZ X3, Else       // if (X3 == 0) {
    ADD  X0, X1, X2     //     X0 = X1 + X2;
    B    Exit           // }
Else:                   // else {
    SUB  X0, X1, X2     //     X0 = X1 - X2;
Exit:                   // }
\end{lstlisting}


\begin{lstlisting}[caption={Simple while loop implementation}]
    ADDI X0, XZR, #10   // X0 = 10;
Loop:
    CBZ  X0, Exit       // while (X0 != 0) {
    SUBI X0, X0, #1     //     X0 -= 1;
    ADD  X1, X2, X0     //     X1 = X2 + X0;
    B    Loop           // }
Exit:
\end{lstlisting}

For a more complete example, the following converts the C code into LEGv8 assembly instructions.

\begin{lstlisting}[caption={C code loop into LEGv8 ASM}]
// C
int i = 0;
int k = 10;
while (save[i] == k) {
    i += 1;
}

// X25: Address of save[]

// Textbook LEGv8 ASM
    ADDI X22, XZR, #0   // EOR X22, X22, X22
    ADDI X14, XZR, #10
Loop:
    LSL  X10, X22, #2
    ADD  X10, X10, X25
    LDUR X11, [X10, #0]
    SUB  X12, X10, X14
    CBNZ X12, Exit
    ADDI X22, X22, #1
    B    Loop
Exit:

// Alternate LEGv8 ASM
    ADD  X10, XZR, X25
    ADDI X11, XZR, #10
Loop:
    LDUR X12, [X10, #0]
    SUB  X12, X12, X11
    CBNZ X12, Exit
    ADDI X10, X10, #8
    B Loop
Exit:
\end{lstlisting}

You can "throw away" the result by writing into \texttt{XZR}, the zero register.

\subsection*{Set Flag Branching}

The conditional branch instruction, \texttt{B.cond}, allows for \texttt{.cond} to be used for signed comparisons, \texttt{EQ} (equals), \texttt{NE} (not equal), \texttt{LT} (less than), \texttt{LE} (less than or equal), \texttt{GT}, or \texttt{GE}.

Also it allows for unsigned comparisons, \texttt{LO}, \texttt{LS} (lower or same), \texttt{HI}, or \texttt{HS} (higher or same).

The flag for comparison is set by instructions like, \texttt{ADDS}, \texttt{ADDIS}, or \texttt{SUBS}, but they are limited in number as they create dependencies that obstruct pipelining execution.

A use case is bounds checking, that is, $0\le x < y$.
\pagebreak

\begin{lstlisting}[caption={Bounds checking shortcut in LEGv8 ASM}]
// Pseudo C
if (X20 >= X11 || X20 < 0) {
    goto Error;
}

// LEGv8 ASM
    SUBS XZR, X20, X11
    B.HS Error
    B    Exit
Error:
    // Error handling
Exit:
\end{lstlisting}

Signed negative numbers look massive when looked at from an unsigned comparison in two's complement, and it also allows for checking if it is below a number (\texttt{X11} in example).

\pagebreak

\section*{Chapter 3}

\subsection*{Floating point representation}

Compromise must be found between the size of the \textit{fraction} and the \textit{exponent}.

\begin{tcolorbox}[
    enhanced,
    attach boxed title to top left={xshift=6mm,yshift=-1.5mm},
    colback=moonstoneblue!20,
    colframe=moonstoneblue,
    colbacktitle=moonstoneblue,
    title=Tradeoff of floating-point representation,
    fonttitle=\bfseries\color{white},
    boxed title style={size=small,colframe=moonstoneblue,sharp corners},
    sharp corners,
    label=box:logic-types,
]
    {\color{moondark}\textbf{Precision}}: Increased by an increase in the size for the \textit{mantissa} or \textit{fraction}. \\
    {\color{moondark}\textbf{Range}}: Increased by an increase in the size for the \textit{exponent}.
\end{tcolorbox}

A LEGv8 implementation of \textbf{floating-point} numbers has \texttt{s} as the sign bit, \texttt{exponent} an 8-bit with bias, and \texttt{fraction} a 23-bit number.

\begin{figure}[htbp]
    \centering
    \fcolorbox{codeFrame}{white}{
       \includegraphics[width=0.7\linewidth]{img/legv8-float.png}
    }
    \caption{\textit{LEGv8 floating-point representation}}
\end{figure}
\vspace{-2em}
$$(-1)^S \times F \times 2^E$$

Range of representation is $2_{\text{ten}} \times 10^{38}$ considering $2^7 (\text{exponent 8-bit rep. divided by 2 for bias}) \approx 38$

Overflow entails the exponent being too large for the \texttt{exponent (E)} field to represent it. Underflow is the same situation but the \texttt{exponent} field representing a negative value.

To reduce the chances of underflow or overflow we use \textbf{double precision}, which is a floating-point value represented in a 64-bit doubleword.

\begin{figure}[htbp]
    \centering
    \fcolorbox{codeFrame}{white}{
       \includegraphics[width=0.85\linewidth]{img/doublep1.png}
    }
    \caption{\textit{LEGv8 double precision floating-point representation}}
\end{figure}

In the case of an overflow/underflow an interrupt (unscheduled disruption call) saves the address of the instruction (to resume after correction) and jumps to a predefined address that deals with the exception.

\subsection*{IEEE 754 Floating-Point Standard Specifications}

The floating-point can always have a leading \texttt{1.0}, as such IEEE assumes it as a hidden bit increasing the representation for the \texttt{fraction} by 1 (23 to 24 in single p., or 52 to 53 in double p.).

\vspace{-2em}
$$(-1)^S \times (1 + F) \times 2^{(\text{Exponent} - \text{Bias})}$$
\vspace{-2em}
$$(-1)^S \times (1 + (s1 \times 2^{-1}) + (s2 \times 2^{-2}) + ...) \times 2^E$$

IEEE 754 has \texttt{NaN} for invalid operations (e.g. $\frac{0}{0}$). Infinities can be represented through the largest exponent representations (+/$-$) instead of interrupting.

The sign bit is the most significant bit to easily process integer comparisons.

For representing the exponent without needing a sign bit, IEEE 754 uses a bias of 127 for single, and 1023 for double, that subtracts from the total possible representation to denote negative and positive exponents.

So for single precision, an exponent of $-1$ is $-1 + 127_{\text{ten}} = 126_{\text{ten}}$, and $+1$ is $1 + 127_{\text{ten}} = 128_{\text{ten}}$

\subsection*{Floating-Point Addition}

\begin{figure}[htbp]
    \centering
    \fcolorbox{codeFrame}{white}{
       \includegraphics[width=0.32\linewidth]{img/floatingpointadd.png}
    }
    \caption{\textit{Floating-point addition logic path}}
\end{figure}
\pagebreak
\section*{Chapter 1}

\subsection*{Design principles}

\subsubsection*{\rightarrow \ \texttt{Design with Moore's Law}}
\vspace{-0.5em}
Moore's Law states that an integrated circuit resources double every 18–24 month.

Thus, computer architects must anticipate where the technology will be when the design finishes rather than design for where it starts.

\subsubsection*{\rightarrow \ \texttt{Abstraction to Simplify Design}}
\vspace{-0.5em}
Abstractions to hide lower-level details to simplify higher-level representations (more productive).

\subsubsection*{\rightarrow \ \texttt{Common case fast}}
\vspace{-0.5em}

If you know what the common case is, enhancing it over the rarer cases is more optimal, as it involves more and is usually easier to enhance.

\subsubsection*{\rightarrow \ \texttt{Performance via Parallelism}}
\vspace{-0.5em}

Computing operations in parallel (at the same time) increases performance over a time period.

\subsubsection*{\rightarrow \ \texttt{Performance via Pipelining}}
\vspace{-0.5em}

Pipelining is a pattern of parallelism that involves increasing the throughput of a certain process (i.e. moving up the chain faster).

\subsubsection*{\rightarrow \ \texttt{Performance via Prediction}}
\vspace{-0.5em}

If you can predict successfully to a degree you can anticipate instead of wait. In some cases it can be more beneficial to guess than to stall when misprediction costs are reasonable.

\subsubsection*{\rightarrow \ \texttt{Memory hierarchy}}
\vspace{-0.5em}

The most expensive and fast memory per bit is at the top, and the opposite is at the bottom. Caches (top) can give the illusion of fast main memory (bottom).

\subsubsection*{\rightarrow \ \texttt{Dependability via Redundancy}}
\vspace{-0.5em}
Systems become dependable when there is redundancy to account for solving failures and failure detections.

\subsection*{Amdahl’s Law}





















% \begin{lstlisting}[caption={Simple SIGINT handler}]
% unsigned char stop = 0;
%
% void signal_handler(int x) {
%     stop = 1;
% }
%
% int main(void) {
%     signal(SIGINT, signal_handler);
%
%     /* Pressing Ctrl-C invokes signal_handler() */
%
%     while (!stop);
%     return 0;
% }
% \end{lstlisting}
%
% This is an example of an image.
%
% \begin{figure}[htbp]
%   \centering
%
%   \fcolorbox{codeFrame}{white}{%
%     \includegraphics[width=0.5\linewidth]{img/graph.png}%
%   }
%
%   \caption{My example image}
%   \label{fig:example}
% \end{figure}



\end{document}
